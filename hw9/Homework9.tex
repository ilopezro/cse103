\documentclass[12pt]{article}
\usepackage[utf8]{inputenc}
\usepackage{graphicx}
\usepackage{setspace}
\usepackage{amsmath}
\usepackage{array}
\usepackage{multirow}
\usepackage[vcentermath]{youngtab}
\newcommand{\rowTwo}[7]{
    row[#1][#2] will look at string $#3#4$. $#3#4$ contains substrings $#3$ and $#4$, which are derived from $#5$ and $#6$ in that order. The cartesian product is $#5#6$. $#5#6$ is found in grammar $\mathcal{A}$ through $#7$. Therefore, row[#1][#2] = $#7$.    
}

\title{CSE103 Homework 9}
\author{Isai Lopez Rodas \\[1em] ilopezro $|$ 1605542}
\date{March 12, 2020}

\begin{document}

\maketitle
\onehalfspacing
\paragraph{Language being referenced in this homework assignment comes from Canvas under Files $\rightarrow$ Handouts $\rightarrow$ CKY Algorithm Notes.pdf\\[1em]
Language $\mathcal{A}$:}
\begin{flalign*}
    &S \rightarrow AB \:|\:BA\:|\:SS\:|\:AC\:|\:BD \\
    &A \rightarrow a \\
    &B \rightarrow b \\
    &C \rightarrow SB \\
    &D \rightarrow SA
\end{flalign*}

\clearpage

\section*{Problem \#1: Is string $babaab \in L(\mathcal{A})$?}

\begin{center}
    \young(6\hfill,5\hfill \hfill,4\hfill \hfill \hfill,3\hfill \hfill \hfill \hfill,2\hfill \hfill \hfill \hfill \hfill,1x\hfill \hfill \hfill \hfill z,\hfill babaab) \\
    Table 1: Initial Unfilled Table
\end{center}

\paragraph{I will denote each row as an array row[$x$][$y$], where $x$ is the row number and $y$ is the index of that array. For example, row[1][0] refers to the bottom left corner marked with an $x$ while row[1][5] is marked with $z$.}
\paragraph{row[1] will contain all possible derivations for the letter right below it.}
\begin{center}
    \young(6\hfill,5\hfill \hfill,4\hfill \hfill \hfill,3\hfill \hfill \hfill \hfill,2\hfill \hfill \hfill \hfill \hfill,1BABAAB,\hfill babaab) \\
    Table 2: Row 1 filled
\end{center}
\paragraph{row[2] contains all possible derivations for substrings of length 2.}
\begin{itemize}
    \item \rowTwo{2}{0}{b}{a}{B}{A}{S}
    \item \rowTwo{2}{1}{a}{b}{A}{B}{S}
    \item \rowTwo{2}{2}{b}{a}{B}{A}{S}
    \item \rowTwo{2}{3}{a}{a}{A}{A}{\emptyset}
    \item \rowTwo{2}{4}{a}{b}{A}{B}{S}
\end{itemize}

\begin{center}
    \young(6\hfill,5\hfill \hfill,4\hfill \hfill \hfill,3\hfill \hfill \hfill \hfill,2SSS\emptyset S,1BABAAB,\hfill babaab) \\
    Table 3: Row 2 filled
\end{center}

\paragraph{row[3] will follow similar procedure as row[2]. We will look at substrings of length 3.}
\begin{itemize}
    \item row[3][0] will look at substring $bab$. This string can be further split up into two substrings, $s_1 = b, ab$ and $s_2 = ba, b$. I will look at $s_1$ first. $b$ can be derived from $B$ and $ab$ can be derived from $S$. The cartesian product is $BS$. $BS$ is not found in grammar $\mathcal{A}$. We will now look at $s_2$. $ba$ is derived from $S$ and $b$ is derived from $B$. Cartesian product $SB$ can be found in grammar $\mathcal{A}$ through $C$. Therefore, since $s_1$ cannot be derived, but $s_2$ can, row[3][0] = $C$.
    \item row[3][1] will look at substring $aba$. This string can be further split up into two substrings, $s_1 = a, ba$ and $s_2 = ab, a$. I will look at $s_1$ first. $a$ can be derived from $A$ and $ba$ can be derived from $S$. Cartersian product $AS$ is not found in grammar $\mathcal{A}$. I will look at $s_2$ now. $ab$ is derived from $S$ and $a$ is derived from $a$. Cartesian product $SA$ is derived through $D$. Therefore, since $s_1$ cannot be derived, but $s_2$ can, row[3][1] = $D$.
    \item row[3][2] will look at substring $baa$. This string produces $s_1 =  b, aa$ and $s_2 = ba, a$. There are no derivations for substring $aa$ in grammar $\mathcal{A}$ as denoted in row[2][3]. For substring $s_2$, we can obtain $ba$ from $S$ and $a$ from $A$. The cartesian product is $SA$, which can be obtain through $D$. Since $s_1$ does not contain a derivation, we do not take that into account and look onto at $s_2$'s derivation. Therefore, row[3][2] = $D$.   
    \item row[3][3] will look at substring $aab$. This string produces $s_1 = a, ab$ and $s_2 = aa, b$. We can discard taking into consideration $s_2$ because it does not have a derivation. $s_1$ can be derived from $A$ and $S$. The cartesian product is $AS$, which is not in the grammar $\mathcal{A}$. Therefore, there are no derivations for any of these substrings so row[3][3] is $\emptyset$.
\end{itemize}

\begin{center}
    \young(6\hfill,5\hfill \hfill,4\hfill \hfill \hfill,3CDD\emptyset,2SSS\emptyset S,1BABAAB,\hfill babaab) \\
    Table 4: Row 3 filled
\end{center}
\clearpage

\paragraph{row[4] will look at substrings of length 4.}
\begin{itemize}
    \item row[4][0] will look at substring $baba$. This string can be broken down into $s_1 = b, aba$, $s_2 = ba, ba$, and $s_3 = bab, a$. For string $s_1$, $b$ is derived from $B$ and $aba$ is derived from $D$. Cartersian product is $BD$. This can be obtained through $S$. For string $s_2$, $ba$ can be obtained from $S$. Since we have two of $ba$, the cartesian product is $SS$, which can be obtained through $S$. For string $s_3$, $bab$ can be obtained through $C$ and $a$ is obtained through $A$. Cartesian product is $CA$, which is not obtained through the grammar at all. Therefore, since $s_1$ and $s_2$ can both be derived by $S$, we say that row[4][0] = $S$. 
    \item row[4][1] will look at substring $abaa$. This string can be split up into: $s_1 = a, baa$, $s_2 = ab, aa$, and $s_3 = aba,a$. For string $s_1$, $a$ is obtained through $A$ and $baa$ is obtained through $D$. Cartersian product $AD$ is not in the grammar. For string $s_2$, $aa$ has no derivation. For string $s_3$, $aba$ is derived through $D$ and $a$ is derived through $A$. Carteisan product $DA$ is also not in the grammar. Therefore, row[4][1] = $\emptyset$.
    \item row[4][2] will look at substring $baab$. This string produces substrings $s_1 = b, aab$, $s_2 = ba, ab$, and $s_3 = baa, b$. For string $s_1$, $aab$ has no derivation. For string $s_2$, $ba$ is obtained through $S$ and $ab$ is obtained through $S$. Carteisan product $SS$ is obtained through $S$. For string $s_3$, $baa$ is obtained through $D$ and $b$ is obtained through $B$. Cartesian product $DB$ is not in the grammar. Therefore, row[4][2] = $S$.
\end{itemize}

\begin{center}
    \young(6\hfill,5\hfill \hfill,4S\emptyset S,3CDD\emptyset,2SSS\emptyset S,1BABAAB,\hfill babaab) \\
    Table 5: Row 4 filled
\end{center}

\paragraph{row[5] will look at substrings of length 5}
\begin{itemize}
    \item row[5][0] will look at substring $babaa$. This string can be split into: $s_1 = b, abaa$, $s_2 = ba, baa$, $s_3 = bab, aa$, and $s_4 = baba, a$. For string $s_1$, $abaa$ has no derivation. For string $s_2$, $ba$ is derived from $S$ and $baa$ is derived from $D$. The cartesian product, $SD$, is not in the grammar. For string $s_3$, $aa$ has no derivation. For string $s_4$, $baba$ is derived through $S$ and $a$ through $A$. Carteisan product, $SA$, is found through $D$. Therefore, row[5][0] = $D$.
    \item row[5][1] will look at substring $abaab$. This string produces substrings: $s_1 = a, baab$, $s_2 = ab, aab$, $s_3 = aba, ab$, and $s_4 = abaa, b$. String $s_1$ can be derived from $A$ and $S$; however, cartesian product $AS$ is not in the grammar. String $s_2$ has no derivation because of $aab$. String $s_3$ can be derived from $D$ and $S$; however, $DS$ is not in the grammar. $s_4$ has no derivation because $abaa$ does not. Therefore, row[5][1] has no derivation. 
\end{itemize}

\begin{center}
    \young(6\hfill,5D\emptyset,4S\emptyset S,3CDD\emptyset,2SSS\emptyset S,1BABAAB,\hfill babaab) \\
    Table 6: Row 5 filled
\end{center}

\paragraph{row[6] will look at substrings of length 6}
\begin{itemize}
    \item row[6][0] will look at string $babaab$. We can have substrings as follows: $s_1 = b, abaab$, $s_2 = ba, baab$, $s_3 = bab, aab$, $s_4 = baba, ab$, and $s_5 = babaa, b$. $s_1$. In $s_1$, $abaab$ has no derivation. In $s_2$, you can derivate the string through $S$ and $S$. Thus the cartesian product $SS$ can be derived from $S$. In $s_3$, $aab$ has no derivation. In $s_4$, you can derive it through $S$ and $S$. The cartesian product is $SS$, which is derived through $S$. In $s_4$, you derive it through $D$ and $B$, but $DB$ is not in the grammar. Since $s_2$ and $s_4$ both get their derivations from $S$, then row[6][0] = $S$. 
\end{itemize}

\begin{center}
    \young(6S,5D\emptyset,4S\emptyset S,3CDD\emptyset,2SSS\emptyset S,1BABAAB,\hfill babaab) \\
    Table 7: Row 6 filled
\end{center}

\paragraph{In conclusion, since row[6] contains the start variable $S$, then the string $babaab \in L(\mathcal{A})$.}
\clearpage

\section*{Problem \#2: Is string $bababb \in L(\mathcal{A})$?}
\paragraph{I will follow the same procedure as I did for problem \#1. For the sake of simplicity, I will give only important and relevant information.}

\begin{center}
    \young(6\hfill,5\hfill \hfill,4\hfill \hfill \hfill,3\hfill \hfill \hfill \hfill,2\hfill \hfill \hfill \hfill \hfill,1\hfill \hfill \hfill \hfill \hfill \hfill,\hfill bababb) \\
    Table 8: Initial Unfilled Table
\end{center}

\paragraph{Again, row[1] contains the derivations of the letters directly below each box.}
\begin{center}
    \young(6\hfill,5\hfill \hfill,4\hfill \hfill \hfill,3\hfill \hfill \hfill \hfill,2\hfill \hfill \hfill \hfill \hfill,1BABABB,\hfill bababb) \\
    Table 9: First Row Filled
\end{center}

\paragraph{The following table are as follows:}
\clearpage
%row 2 problem 2
\begin{table}[h]
    \centering
    \begin{tabular}{|>{\centering\arraybackslash}p{1.6cm}|>{\centering\arraybackslash}p{1cm}|>{\centering\arraybackslash}p{2cm}|>{\centering\arraybackslash}p{2cm}|>{\centering\arraybackslash}p{2.5cm}|>{\centering\arraybackslash}p{3cm}|}
    \hline
                                & s                 & substring of s & derivated from & Cartesian Product & In grammar? Where? \\ \hline
    \multirow{2}{*}{row[2][0]}  &\multirow{2}{*}{ba}&     b           &     B           & \multirow{2}{*}{BA} & \multirow{2}{*}{\{S\}}  \\ \cline{3-4}
                                &                   &       a         &      A          &                     &                      \\ \hline
    \multirow{2}{*}{row[2][1]}  &\multirow{2}{*}{ab} &       a         &     A           & \multirow{2}{*}{AB} & \multirow{2}{*}{\{S\}}  \\ \cline{3-4}
                                &                    &          b      &       B         &                   &                    \\ \hline 
    \multirow{2}{*}{row[2][2]}  &\multirow{2}{*}{ba}&     b           &     B           & \multirow{2}{*}{BA} & \multirow{2}{*}{\{S\}}  \\ \cline{3-4}
                                &                   &       a         &      A          &                     &                      \\ \hline
    \multirow{2}{*}{row[2][3]}  &\multirow{2}{*}{ab} &       a         &     A           & \multirow{2}{*}{AB} & \multirow{2}{*}{\{S\}}  \\ \cline{3-4}
                                &                    &          b      &       B         &                   &                    \\ \hline    
    \multirow{2}{*}{row[2][4]}  &\multirow{2}{*}{bb} &       b         &     B           & \multirow{2}{*}{BB} & \multirow{2}{*}{$\emptyset$}  \\ \cline{3-4}
                                &                    &       b         &       B         &                   &                    \\ \hline       
    \end{tabular}
\end{table}

%row 3 problem 2
\begin{table}[h]
    \centering
    \begin{tabular}{|>{\centering\arraybackslash}p{1.7cm}|>{\centering\arraybackslash}p{1cm}|>{\centering\arraybackslash}p{2cm}|>{\centering\arraybackslash}p{2cm}|>{\centering\arraybackslash}p{2.5cm}|>{\centering\arraybackslash}p{3cm}|}
    \hline
                                & s                   & substring of s & derivated from  & Cartesian Product            & In grammar? Where?       \\ \hline
    \multirow{2}{*}{row[3][0]}  &\multirow{2}{*}{bab} &     b, ab      &     B, S        &         \{BS\}               & \multirow{2}{*}{\{C\}}    \\ \cline{3-5}
                                &                     &     ba, b      &     S, B        &         \{SB\}               &                           \\ \hline
    \multirow{2}{*}{row[3][1]}  &\multirow{2}{*}{aba} &     a, ba      &     A, S        &         \{AS\}               & \multirow{2}{*}{\{D\}}  \\ \cline{3-5}
                                &                     &     ab, a      &     S, A        &         \{SA\}               &                            \\ \hline 
    \multirow{2}{*}{row[3][2]}  &\multirow{2}{*}{bab} &     b, ab      &     B, S        &         \{BS\}               & \multirow{2}{*}{\{C\}}  \\ \cline{3-5}
                                &                     &     ba, b      &     S, B        &         \{SB\}               &                      \\ \hline
    \multirow{2}{*}{row[3][3]}  &\multirow{2}{*}{abb} &     a, bb      &  A, $\emptyset$ &       $\emptyset$            & \multirow{2}{*}{\{C\}}  \\ \cline{3-5}
                                &                     &     ab, b      &     S, B        &         \{SB\}               &                    \\ \hline    
    \end{tabular}
\end{table}

\clearpage

%row 4 problem 2
\resizebox{\textwidth}{!}{%
    \centering
    \begin{tabular}{|c|c|c|c|c|c|c|}
    \hline
                               &         s             & substring of s & derivated from & Cartesian Product & In grammar?        & Where?                           \\ \hline
    \multirow{3}{*}{row[4][0]} & \multirow{3}{*}{baba} & b, aba         & B, D           &    \{BD\}           &  \{S\}           & \multirow{3}{*}{\{S\}}           \\ \cline{3-6}
                               &                       & ba, ba         & S, S           &    \{SS\}           &  \{S\}           &                                  \\ \cline{3-6}
                               &                       & bab, a         & C, A           &    \{CA\}           &  \{$\emptyset$\} &                                  \\ \hline
    \multirow{3}{*}{row[4][1]} & \multirow{3}{*}{abab} & a, bab         & A, C           &    \{AC\}           &  \{S\}           & \multirow{3}{*}{\{S\}}           \\ \cline{3-6}
                               &                       & ab, ab         & S, S           &    \{SS\}           &  \{S\}           &                                  \\ \cline{3-6}
                               &                       & aba, b         & D, B           &    \{$\emptyset$\}  &  \{$\emptyset$\} &                                  \\ \hline
    \multirow{3}{*}{row[4][2]} & \multirow{3}{*}{babb} & b, abb         & B, C           &    \{BC\}           &  \{$\emptyset$\} & \multirow{3}{*}{\{$\emptyset$\}} \\ \cline{3-6}
                               &                       & ba, bb         & S, $\emptyset$ &    \{$\emptyset$\}  &  \{$\emptyset$\} &                                  \\ \cline{3-6}
                               &                       & bab, b         & C, A           &    \{CA\}           &  \{$\emptyset$\} &                                   \\ \hline
    \end{tabular}%
}
\\[2em]

%row 5 problem 2
\resizebox{\textwidth}{!}{%
    \centering
    \begin{tabular}{|c|c|c|c|c|c|c|}
    \hline
                               &       s                & substrings of s & derivated from  & Carteisan Product & In Grammar?     & Final                  \\ \hline
    \multirow{4}{*}{row[5][0]} & \multirow{4}{*}{babab} &  b, abab        &  B, S           &   \{BS\}          & \{$\emptyset$\} & \multirow{4}{*}{C} \\ \cline{3-6}
                               &                        &  ba, bab        &  S, C           &   \{SC\}          & \{$\emptyset$\} &                   \\ \cline{3-6}
                               &                        &  bab, ab        &  C, S           &   \{CS\}          & \{$\emptyset$\} &                   \\ \cline{3-6}
                               &                        &  baba, b        &  S, B           &   \{SB\}          & \{C\}           &                   \\ \hline
    \multirow{4}{*}{row[5][1]} & \multirow{4}{*}{ababb} &  a, babb        &  A, $\emptyset$ &   \{$\emptyset$\} & \{$\emptyset$\} & \multirow{4}{*}{C} \\ \cline{3-6}
                               &                        &  ab, abb        &  S, C           &   \{SC\}          & \{$\emptyset$\} &                   \\ \cline{3-6}
                               &                        &  aba, bb        &  D, $\emptyset$ &   \{$\emptyset$\} & \{$\emptyset$\} &                   \\ \cline{3-6}
                               &                        &  abab, b        &  S, B           &   \{SB\}          & \{C\}           &                   \\ \hline
    \end{tabular}%
}
\paragraph{\\}

%row 6 problem 2
\resizebox{\textwidth}{!}{%
    \centering
    \begin{tabular}{|c|c|c|c|c|c|c|}
    \hline
                               &       s                 & substrings of s & derivated from  & Carteisan Product & In Grammar?     & Final                  \\ \hline
    \multirow{5}{*}{row[6][0]} & \multirow{5}{*}{bababb} & b, ababb        &  B, C           & \{BC\}            & \{$\emptyset$\} & \multirow{5}{*}{\{$\emptyset$\}} \\ \cline{3-6}
                               &                         & ba, babb        &  S, $\emptyset$ & \{$\emptyset$\}   & \{$\emptyset$\} &                   \\ \cline{3-6}
                               &                         & bab, abb        &  C, C           & \{CC\}            & \{$\emptyset$\} &                   \\ \cline{3-6}
                               &                         & baba, bb        &  S, $\emptyset$ & \{$\emptyset$\}   & \{$\emptyset$\} &                   \\ \cline{3-6}
                               &                         & babab, b        &  C, B           & \{CB\}            & \{$\emptyset$\} &                   \\ \hline
    \end{tabular}%
}

\begin{center}
    \young(6\emptyset,5CC,4SS\emptyset,3CDCC,2SSSS\emptyset,1BABABB,\hfill bababb) \\
    Table 10: Final Table
\end{center}
\paragraph{In conclusion, since row[6] does not contain start variable $S$, then string $bababb \notin L(\mathcal{A})$.}


\end{document}